\section{Discussion}
This study hypothesizes that in low-dimensional spaces during a WM task in humans, hippocampal neurons exhibit WM-task dependent NTs, primarily during SWR periods. Initially, multi-unit spikes in the MTL regions were projected onto three-dimensional spaces during a Sternberg task using GPFA (Figure~\ref{fig:01}D--E \& Figure~\ref{fig:02}A). The NT distances across WM phases ($\mathrm{\lVert g_{F}g_{E} \rVert}$, $\mathrm{\lVert g_{F}g_{M} \rVert}$, $\mathrm{\lVert g_{F}g_{R} \rVert}$, $\mathrm{\lVert g_{E}g_{M} \rVert}$, $\mathrm{\lVert g_{E}g_{R} \rVert}$, and $\mathrm{\lVert g_{M}g_{R} \rVert}$) were significantly larger in the hippocampus compared to the EC and amygdala (Figure~\ref{fig:02}C--E). Also, in the hippocampus, the NT distance between the encoding and retrieval phases ($\mathrm{\lVert g_{F}g_{E} \rVert}$) positively correlated with memory load (Figure~\ref{fig:03}C--D). The hippocampal NT transiently expanded during SWRs (Figure~\ref{fig:05}). Lastly, the hippocampal NT alternated between encoding and retrieval states, transitioning from encoding to retrieval states during SWR events (Figure~\ref{fig:07}). These findings explain aspects of hippocampal neural activity during a WM task in humans and offer new insights into SWRs as a state-switching manifestation in hippocampal neural states.\\
\indent
The longer disntace of NTs across the four phases in the hippocampus indicates dynamic and responsive neural activity in the hippocampus during the WM task. This observation corroborates previous studies indicating hippocampal persistent firing during the maintenance phase \cite{kaminski_persistently_2017, kornblith_persistent_2017, faraut_dataset_2018, boran_persistent_2019}. In addition to existing literature, the current study, through the application of GPFA to multi-unit activity during a one-second level resolution of the WM task, revealed that the NT in low-dimensional space presented a memory-load dependency between the encoding and retrieval phases ($\mathrm{\lVert g_{E}g_{R} \rVert}$) (Figure~\ref{fig:03}). Interestingly, this dependency was not identified in other phase combinations, suggesting that the encoding and retrieval states expanded in opposite directions in response to the WM load. Overall, these findings not only reinforce the association of the hippocampus with WM processing in humans but also propose a new concept of "hippocampal neural fluctuation between encoding and retrieval states".\\
\indent
Our analysis focused on the putative CA1 regions (Figure~\ref{fig:04}) to enhance the validity of the recording site for SWR detection, a task that is challenging in human studies due to the frequent unavailability of post-mortem histology. This criterion is supported by accumulated evidence. For instance, SWRs synchronize with spike bursts of interneuron and pyramidal neuron \cite{buzsaki_two-stage_1989, quyen_cell_2008, royer_control_2012, hajos_input-output_2013}, potentially within a 50 $\mu$m radius of the recording site \cite{schomburg_spiking_2012}. Additionally, we identified increased incidence of SWRs during the first 0--400 ms of the retrieval phase (Figure~\ref{fig:04}D). This finding aligns with previous reports of heightened SWR occurrence preceding spontaneous verbal recall \cite{norman_hippocampal_2019, norman_hippocampal_2021}, extending our understanding to a triggered retrieval condition. Moreover, the log-normal distributions of both SWR duration and ripple band peak amplitude observed in this study (Figure~\ref{fig:04}C \& E) coincide with the consensus in this field \cite{liu_consensus_2022}. Therefore, these results support the electrode placement and detected SWRs in this study. One could argue that the NT distance increase from $O$ during SWRs (Figure~\ref{fig:05}) may be artificially inflated towards higher values due to channel selection using UMAP clustering on spike counts. However, this potential bias does not affect the direction of NT, the memory-load dependency, nor the WM task dependency identified in this study.\\
\indent
Interestingly, NT directions alternated between encoding and retrieval states during the retrieval phase of both baseline and SWR periods in a task-dependent manner (Figure~\ref{fig:07}C \& D). Additionally, the balance of this fluctuation transitioned from encoding to retrieval state during SWR events (Figure~\ref{fig:07} E \& F). These results align with previous studies on the role of SWR in memory retrieval \cite{norman_hippocampal_2019, norman_hippocampal_2021}. Our findings demonstrate that, during a WM task in humans, (i) neuronal fluctuation between encoding and retrieval states occurs, and (ii) SWR events serve as indicators of the transition in hippocampal neural states from encoding to retrieval.\\
\indent
Moreover, our study noted differences specific to the WM-task type between encoding- and retrieval-SWRs (Figure~\ref{fig:07}E--F). Notably, opposing movements of encoding-SWR (eSWR) and retrieval-SWR (rSWR) were not observed in Match IN task but were apparent in Mismatch OUT task. These results might be explained by the memory engram theory \cite{liu_optogenetic_2012}; Match In task presented participants with previously shown letters, while Mismatch OUT task introduced a new letter absent in the encoding phase. This explanation underscores the significant role of SWR in human cognitive processes.\\
\indent
In conclusion, this study illustrates that during a WM task in humans, hippocampal activity fluctuate between encoding and retrieval states in a manner dependent on memory load and task, with notable transition from encoding to retrieval states during SWR events. These findings offer novel insights into the neural correlates and functionality of working memory within the hippocampus.
\label{sec:discussion}

% The application of log-transformation to NT distances influences the correlation analysis in this study (Figure~\ref{fig:03}C \& D), although other statistical tests performed are nonparametric and not affected by the log-transformation. Moreover, The use of log-transformation is justified from several perspectives. First, log-normal distributions are prevalent in the central nervous system, including in the firing rate, the weight and conductance of synpses \cite{ikegaya_interpyramid_2013, buzsaki_log-dynamic_2014} in the hippocampus, and SWR amplitude and duration \cite{liu_consensus_2022}. Furthermore, GPFA linearly projects neural spike data into a latent space, preserving the temporal structure with Gaussian processes. Therefore, if the input data, spike counts in our case, is log-normally distributed, the latent GPFA space should, ideally, reflect the log-normality. Lastly, the empirical observation of the NT distances, as seen in Figure~\ref{fig:01}, supports the efficacy of our method in detecting subtle differences during baseline periods. Accordingly, the correlation analysis is the only section potentially affected by the use of log-normalization; but this approach will be justified from previous works.
