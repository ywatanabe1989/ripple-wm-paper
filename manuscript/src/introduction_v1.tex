\section{Introduction}
Working memory (WM) is crucial in everyday life; however, its neural mechanism has yet to be fully elucidated. Specifically, the hippocampus's involvement in WM processing, a pivotal region for memory, is the subject of ongoing research \cite{scoville_loss_1957, squire_legacy_2009, boran_persistent_2019, kaminski_persistently_2017, kornblith_persistent_2017, faraut_dataset_2018, borders_hippocampus_2022, li_functional_2023, dimakopoulos_information_2022}. Understanding the hippocampus’ role in working memory is instrumental in deepening our comprehension of cognitive processes and could potentially enhance cognitive abilities.
\\
\indent
Current evidence suggests that transient, synchronized oscillations, termed sharp-wave ripples (SWRs) \cite{buzsaki_hippocampal_2015}, are associated with a variety of cognitive functions. SWRs have traditionally been linked with long-term memory functions such as memory replay \cite{wilson_reactivation_1994, nadasdy_replay_1999, lee_memory_2002, diba_forward_2007, davidson_hippocampal_2009}, memory consolidation \cite{girardeau_selective_2009, ego-stengel_disruption_2010, fernandez-ruiz_long-duration_2019, kim_corticalhippocampal_2022}, memory recall \cite{wu_hippocampal_2017, norman_hippocampal_2019, norman_hippocampal_2021}, and neural plasticity \cite{behrens_induction_2005, norimoto_hippocampal_2018}. However, only a subset of studies has investigated the role of SWRs in WM tasks \cite{jadhav_awake_2012, sasaki_dentate_2018}. This gap in our understanding motivates the current study to further investigate the potential involvement of SWRs in WM, particularly given their fundamental computational manifestation in hippocampal processing.
\\
\indent
Recent studies have found that low-dimensional representations in hippocampal neurons can explain WM task performances. Specifically, the firing patterns of place cells \cite{okeefe_hippocampus_1971, okeefe_place_1976, ekstrom_cellular_2003, kjelstrup_finite_2008, harvey_intracellular_2009}, found in the hippocampus, have been identified within dynamic, nonlinear three-dimensional hyperbolic spaces in rats \cite{zhang_hippocampal_2022}. Additionally, grid cells in the entorhinal cortex (EC), which is the main pathway to the hippocampus \cite{naber_reciprocal_2001, van_strien_anatomy_2009, strange_functional_2014}, exhibited a toroidal geometry during exploration in rats \cite{gardner_toroidal_2022}.
\\
\indent
However, these existing studies predominantly focus on spatial navigation in rodents, presenting several limitations. First, the temporal resolution of navigation tasks is insufficient, obscuring the precise timing of memory acquisition and recall. Second, the presence of noise in signals recorded during rodent movement complicates the detection of SWRs \cite{Watanabe_2021}. Third, the generalization to humans and tasks other than spatial navigation remains unclear. Given these limitations, it is crucial to explore SWRs in a controlled, less noisy environment to better understand their potential role in WM tasks in humans.
\\
\indent
Considering these factors, this study investigates the hypothesis that hippocampal neurons in humans exhibit low-dimensional neural trajectories (NTs) that depend on WM load, particularly during SWR periods. To test this hypothesis, we employed a dataset of patients performing an eight-second Sternberg task (1 s for fixation, 2 s for encoding, 3 s for maintenance, and 2 s for retrieval) with high temporal resolution. Intracranial electroencephalography (iEEG) signals within the medial temporal lobe (MTL) were recorded for these patients \cite{boran_dataset_2020}. To investigate low-dimensional NTs, we utilized Gaussian-process factor analysis (GPFA), an established method for analyzing neural population dynamics \cite{yu_gaussian-process_2009, churchland_stimulus_2010, lin_functional_2011, churchland_neural_2012, ecker_state_2014, kao_single-trial_2015, gallego_neural_2017, wei_orderly_2019, kim_corticalhippocampal_2023}.
\label{sec:introduction}