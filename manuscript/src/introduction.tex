\section{Introduction}
Working memory (WM), serving as a key player in cognitive abilities, underpins our daily activities and relations with the world, from basic perceptual decision making to sophisticated cognitive operations. One remarkable component in the neural mechanisms of WM is the hippocampus, an area identified as being crucial for various forms of memory \cite{scoville_loss_1957, squire_legacy_2009, boran_persistent_2019, kaminski_persistently_2017, kornblith_persistent_2017, faraut_dataset_2018, borders_hippocampus_2022, li_functional_2023, dimakopoulos_information_2022}. Unraveling the role and contributions of the hippocampus within the realm of WM informs our understanding of the cognitive dynamics underpinning everyday functionality. This knowledge may ultimately foster the enhancement of cognitive performance and the development of interventions for memory-related disorders.
\\
\indent
Hippocampal networks yield transient, synchronized oscillations known as sharp-wave ripples (SWRs), which have been found to replay sequences of recent and prospective memory traces during spatial navigation tasks \cite{foster_reverse_2006, karlsson_awake_2009, carr_hippocampal_2011, pfeiffer_hippocampal_2013}. Moreover, functional correlations between awake SWRs and spatial navigation WM performance over multi-day scales have been elucidated by selective SWR suppressions \cite{girardeau_selective_2009, jadhav_awake_2012, singer_hippocampal_2013}, and prolongation events \cite{fernandez-ruiz_long-duration_2019}, as well as functional lesioning in a subregion of the hippocampus \cite{sasaki_dentate_2018}. Certain studies have emphasized the coordination of SWRs with other oscillation components in the facilitation of WM processing \cite{tamura_hippocampal-prefrontal_2017, daume_control_2024}. Furthermore, SWR events that occur seconds before memory recall proffer a notion of their essentiality for effective execution of WM tasks \cite{wu_hippocampal_2017, norman_hippocampal_2019, norman_hippocampal_2021}. Despite these preliminary insights, our comprehensive understanding of SWRs and their temporal relationship with WM processes remains largely incomplete.
\\
\indent
One noteworthy limitation in the current body of research is predicated on the use of rodent navigation tasks, wherein the temporal attributes of the task were not sufficiently granular to discern the exact timings of memory acquisition, retrieval, and decision-making processes. Furthermore, the detection of SWRs during predominantly immobile periods in rodents \cite{foster_reverse_2006, karlsson_awake_2009, carr_hippocampal_2011, pfeiffer_hippocampal_2013, jadhav_awake_2012, singer_hippocampal_2013, fernandez-ruiz_long-duration_2019}, likely due to potential contamination from electromyographic noise \cite{Watanabe_2021}, limits the extrapolation of experimental findings to subjectively equivalent functionality of WM in humans. It is thus imperative to explore the relationship between SWRs and human WM in a noise-controlled experimental setup with a temporally precise measurement of WM events.
\\
\indent
The present study hypothesizes that human hippocampal neurons manifest low-dimensional neural trajectories (NTs) that fluctuate with WM load, particularly during SWR periods. The emphasis on NTs is derived from the imperative to comprehend the continuous, dynamic representation of neurons and the facilitation of visualization and comprehension. To evaluate this hypothesis, a human WM dataset characterized by high temporal resolution --- 1 s for fixation, 2 s for encoding, 3 s for maintenance, and 2 s for retrieval --- \cite{boran_dataset_2020} was employed. While this dataset has been employed in several studies \cite{li_functional_2023, sheybani_wake_2023, li_anteriorposterior_2022, li_thetaalpha_2024, ye_phase-amplitude_2022, cocina_spiking_2022, dimakopoulos_information_2022}, this is the first study to investigate SWRs in this particular dataset.
\\
\indent
To aptly analyze low-dimensional dynamics, dimensionality reduction techniques were implemented. Interestingly, recent findings have shown the presence of dynamic, nonlinear low-dimensional spaces within the firing patterns of hippocampal \cite{zhang_hippocampal_2022} and entorhinal cortex (EC) neurons in rodents \cite{gardner_toroidal_2022}. Accordingly, we applied Gaussian-process factor analysis (GPFA), a proven methodology for neural population dynamics analyses \cite{yu_gaussian-process_2009, churchland_stimulus_2010, lin_functional_2011, churchland_neural_2012, ecker_state_2014, kao_single-trial_2015, gallego_neural_2017, wei_orderly_2019, kim_corticalhippocampal_2023}, on intracranial electroencephalography (iEEG) signals recorded from the medial temporal lobe, including the hippocampus, of patients performing a structured WM task.
\label{sec:introduction}
% \\
% \indent
% The current study offers an in-depth view into the characteristics and behaviors of low-dimensional neural trajectories within the hippocampal region, allowing us to examine how WM load modulates these trajectories and observe the role of SWRs in these complex dynamics.
% \\
% \indent
% Our findings contribute a pioneering perspective to our understanding of hippocampal dynamics during WM tasks and illuminate critical aspects of the cognitive processing of these tasks. They might also potentially inform the future developments in memory enhancement strategies and interventions for memory deficiencies.




% Working memory (WM) plays a vital role in everyday life, yet its neural mechanism remains to be fully elucidated. One region implicated in WM processing is the hippocampus, a critical area for memory \cite{scoville_loss_1957, squire_legacy_2009, boran_persistent_2019, kaminski_persistently_2017, kornblith_persistent_2017, faraut_dataset_2018, borders_hippocampus_2022, li_functional_2023, dimakopoulos_information_2022}. Understanding the role of the hippocampus in working memory is key to enhancing our understanding of cognitive processes and may potentially influence cognitive abilities.
% \\
% \indent
% During WM tasks, transient, synchronized oscillations known as sharp-wave ripples (SWRs) \cite{buzsaki_hippocampal_2015} are observed to replay recent and future memory traces \cite{foster_reverse_2006, karlsson_awake_2009, carr_hippocampal_2011, pfeiffer_hippocampal_2013}. Moreover, awake SWRs are functionally associated with WM performance in several-day scales, as demonstrated by selective SWR suppressions \cite{girardeau_selective_2009, jadhav_awake_2012, singer_hippocampal_2013} and prolongation \cite{fernandez-ruiz_long-duration_2019}, as well as functional lesions in the dentate gyrus \cite{sasaki_dentate_2018}. Some studies also underscore the role of SWRs in WM performance, coordinated with other oscillation components \cite{tamura_hippocampal-prefrontal_2017, daume_control_2024}. In addition to these findings, SWRs are observed a few seconds before memory recall \cite{wu_hippocampal_2017, norman_hippocampal_2019, norman_hippocampal_2021}, suggesting their necessity for successful WM processing.
% \\
% \indent
% However, these behavioral memory functions have primarily been investigated using rodent navigation tasks that rely on place cells \cite{okeefe_hippocampus_1971, okeefe_place_1976, ekstrom_cellular_2003, kjelstrup_finite_2008, harvey_intracellular_2009}. Thus, the temporal resolution of WM tasks was not sufficient to distinguish exact timings of memory acquisition, retrieval, and decision-making processes. Furthermore, in rodent navigation tasks, SWRs are detected predominantly during immobility (\textit{e.g.}, when the animal moves at a speed of less than 4 cm/s at reward locations) \cite{foster_reverse_2006, karlsson_awake_2009, carr_hippocampal_2011, pfeiffer_hippocampal_2013, jadhav_awake_2012, singer_hippocampal_2013, fernandez-ruiz_long-duration_2019}, potentially due to contamination from electromyographical noise \cite{Watanabe_2021}, leading to a lack of generalization to human WM functionality. These limitations prompted us to investigate the correlation between SWRs and the exact timings of WM using a high temporal resolution dataset in humans in a controlled, less noisy experimental setting.
% \\
% \indent
% In order to better understand the dynamic computation of hippocampus, dimensionality reduction techniques will be beneficial. In fact, the firing patterns of hippocampal place cells have been identified within dynamic, nonlinear three-dimensional hyperbolic spaces in rats \cite{zhang_hippocampal_2022}. Additionally, grid cells in the entorhinal cortex (EC), which is the main gateway of the hippocampus \cite{naber_reciprocal_2001, van_strien_anatomy_2009, strange_functional_2014}, exhibited a toroidal geometry during exploration in rats \cite{gardner_toroidal_2022}.
% \\
% \indent
% Considering these factors, this study hypothesize that hippocampal neurons in humans exhibit low-dimensional neural trajectories (NTs) that depend on WM load, particularly during SWR periods. To test this hypothesis, we employed a dataset of patients performing an eight-second Sternberg task with high temporal resolution (1 s for fixation, 2 s for encoding, 3 s for maintenance, and 2 s for retrieval) \cite{boran_dataset_2020}. Intracranial electroencephalography (iEEG) signals within the medial temporal lobe (MTL) were recorded for these patients. To investigate low-dimensional NTs, we utilized Gaussian-process factor analysis (GPFA), an established method for analyzing neural population dynamics \cite{yu_gaussian-process_2009, churchland_stimulus_2010, lin_functional_2011, churchland_neural_2012, ecker_state_2014, kao_single-trial_2015, gallego_neural_2017, wei_orderly_2019, kim_corticalhippocampal_2023}.
% \label{sec:introduction}