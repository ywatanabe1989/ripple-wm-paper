\section{Discussion}
This study hypothesizes that in low-dimensional spaces during a WM task in humans, hippocampal neurons form unique NTs, primarily during SWR periods. Initially, multi-unit spikes in the MTL regions were projected onto three-dimensional spaces during a Sternberg task using GPFA (Figure~\ref{fig:01}D--E \& Figure~\ref{fig:02}A). The NT distances across WM phases ($\mathrm{\lVert g_{F}g_{E} \rVert}$, $\mathrm{\lVert g_{F}g_{M} \rVert}$, $\mathrm{\lVert g_{F}g_{R} \rVert}$, $\mathrm{\lVert g_{E}g_{M} \rVert}$, $\mathrm{\lVert g_{E}g_{R} \rVert}$, and $\mathrm{\lVert g_{M}g_{R} \rVert}$) were significantly larger in the hippocampus compared to the EC and amygdala (Figure~\ref{fig:02}C--E), indicating dynamic and responsive neural activity in the hippocampus during the WM task. Also, in the hippocampus, the NT distance between the encoding and retrieval phases ($\mathrm{\lVert g_{F}g_{E} \rVert}$) correlated positively with memory load (Figure~\ref{fig:03}C--D), reflecting WM processing. The hippocampal neural NT transiently expanded during SWRs (Figure~\ref{fig:05}). Lastly, the hippocampal neural NT alternated between encoding and retrieval states, transitioning from encoding to retrieval during SWR events (Figure~\ref{fig:07}). These findings explain aspects of hippocampal neural activity during a WM task in humans and offer new insights into SWRs as a state-switching element in hippocampal neural states.

The distance of the neural NT across the phases was significantly longer in the hippocampus compared to the EC and amygdala, even when considering the distance from $O$ in these regions (Figure~\ref{fig:02}C--E). This establishes the involvement of the hippocampus in the WM task, corroborating previous studies indicating hippocampal persistent firing during the maintenance phase \cite{boran_persistent_2019} \cite{kaminski_persistently_2017} \cite{kornblith_persistent_2017} \cite{faraut_dataset_2018}. However, in the present study, applying GPFA to multi-unit activity during a one-second level resolution of the WM task revealed that the neural NT in low-dimensional space presented a memory-load dependency between the encoding and retrieval phases, denoted as $\mathrm{\lVert g_{E}g_{R} \rVert}$ (Figure~\ref{fig:03}). These findings support the association of the hippocampus with WM processing.

Our analysis focused on putative CA1 regions (Figure~\ref{fig:04}), is supported by several factors. This specific focus results from established observations that SWRs synchronize with interneuron and pyramidal neuron spike bursts \cite{buzsaki_two-stage_1989} \cite{quyen_cell_2008} \cite{royer_control_2012} \cite{hajos_input-output_2013}, potentially within a 50 $\mu$m radius of the recording site \cite{schomburg_spiking_2012}. Furthermore, we identified an increased incidence of SWRs during the first 0--400 ms of the retrieval phase (Figure~\ref{fig:04}D). This finding aligns with previous reports of heightened SWR occurrence preceding spontaneous verbal recall \cite{norman_hippocampal_2019} \cite{norman_hippocampal_2021}, supporting our results under a triggered retrieval condition. The observed log-normal distributions of both SWR duration and ripple band peak amplitude in this study (Figure~\ref{fig:04}C \& E) coincide with the current consensus in this field \cite{liu_consensus_2022}. Consequently, our decision to limit recording sites to putative CA1 regions likely contributed to improving the precision, or true positive rate, of SWR detection. Although, the NT distance increase from $O$ during SWRs (Figure~\ref{fig:05}) might be artificially inflated towards higher values due to channel selection, this potential bias does not substantially challenge our main findings.

Interestingly, during the retrieval phase, NT directions alternated between encoding and retrieval states during both baseline and SWR periods in a task-dependent manner (Figure~\ref{fig:07}C \& D). Additionally, the balance of this fluctuation transitioned from encoding to retrieval state during SWR events (Figure~\ref{fig:07} E \& F). These results align with previous studies on the role of SWR in memory retrieval \cite{norman_hippocampal_2019} \cite{norman_hippocampal_2021}. Our findings suggest that (i) neuronal oscillation between encoding and retrieval states occurs during a WM task and (ii) SWR events serve as indicators of the transition in hippocampal neural states from encoding to retrieval during a WM task.

Moreover, our study noted differences specific to the WM-task type between encoding- and retrieval-SWRs (Figure~\ref{fig:07}E--F). Notably, opposing movements of encoding-SWR (eSWR) and retrieval-SWR (rSWR) were not observed in Match IN task but were apparent in Mismatch OUT task. Memory engram theory \cite{liu_optogenetic_2012} might explain these observations: Match In task presented participants with previously shown letters, while Mismatch OUT task introduced a new letter absent in the encoding phase. This explanation underscores the significant role of SWR in human cognitive processes.

The application of log-transformation to NT distances influences the correlation analysis in this study (Figure~\ref{fig:03}C \& D), although other statistical tests performed are nonparametric and not affected by the log-transformation. Moreover, The use of log-transformation is justified from several perspectives. First, log-normal distributions are prevalent in the central nervous system, including in the firing rate, the weight and conductance of synpses \cite{buzsaki_log-dynamic_2014} \cite{ikegaya_interpyramid_2013} in the hippocampus, and SWR amplitude and duration \cite{liu_consensus_2022}. Furthermore, GPFA linearly projects neural spike data into a latent space, preserving the temporal structure with Gaussian processes. Therefore, if the input data, spike counts in our case, is log-normally distributed, the latent GPFA space should, ideally, reflect the log-normality. Lastly, the empirical observation of the NT distances, as seen in Figure~\ref{fig:01}, supports the efficacy of our method in detecting subtle differences during baseline periods. Accordingly, the correlation analysis is the only section potentially affected by the use of log-normalization; but this approach will be justified from previous works.

In conclusion, this study illustrates that during a WM task in humans, hippocampal activity fluctuate between encoding and retrieval states, uniquely transitioning from encoding to retrieval during SWR events. These findings offer novel insights into the neural correlates and functionality of working memory within the hippocampus.
\label{sec:discussion}