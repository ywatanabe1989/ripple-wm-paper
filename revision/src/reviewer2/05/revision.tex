{
\color{revision_color}
\revEditor{}
%%%%%%%%%%%%%%%%%%%%%%%%%%%%%%%%%%%%%%%%%%%%%%%%%%%%%%%%%%%%%%%%%%%%%%%%%%%%%%%%
\subsection{Defining SWRs from \DIFdelbegin \DIFdel{putative hippocampal }\DIFdelend \DIFaddbegin \DIFadd{Putative Hippocampal }\DIFaddend CA1 \DIFdelbegin \DIFdel{regions}\DIFdelend \DIFaddbegin \DIFadd{Regions Using UMAP Clustering}\DIFaddend }
Potential SWRs were differentiated from SWR candidates in putative CA1 (cornu Ammonis 1) regions. \DIFdelbegin \DIFdel{These regions were initially defined as follows: $\textrm{SWR}^+/\textrm{SWR}^-$ }\DIFdelend \DIFaddbegin \DIFadd{The definition of putative CA1 regions was as follows. First, $\textrm{SWR}^+$ and $\textrm{SWR}^-$ }\DIFaddend candidates in the hippocampus were projected into a two-dimensional space \DIFdelbegin \DIFdel{based on overlapping spike counts per unit }\DIFdelend using a supervised \DIFdelbegin \DIFdel{method, UMAP (}\DIFdelend \DIFaddbegin \DIFadd{clustering method, }\DIFaddend Uniform Manifold Approximation and Projection \DIFaddbegin \DIFadd{(UMAP}\DIFaddend ) \cite{mcinnes_umap_2018}. \DIFaddbegin \DIFadd{The input features for this projection were the spike counts per unit during the period of $\textrm{SWR}^+$ or $\textrm{SWR}^-$ candidates. }\DIFaddend Clustering validation was performed by calculating the silhouette score \cite{rousseeuw_silhouettes_1987} from clustered \DIFdelbegin \DIFdel{samples}\DIFdelend \DIFaddbegin \DIFadd{sample points in the corresponding two-dimensional space}\DIFaddend . Regions in the hippocampus \DIFdelbegin \DIFdel{, which }\DIFdelend \DIFaddbegin \DIFadd{that }\DIFaddend scored above 0.6 on average across sessions ($75^{th}$ percentile) \DIFdelbegin \DIFdel{, }\DIFdelend were identified as putative CA1 regions\DIFdelbegin \DIFdel{, resulting }\DIFdelend \DIFaddbegin \DIFadd{. This process resulted }\DIFaddend in the identification of five electrode positions from five patients.
%%%%%%%%%%%%%%%%%%%%%%%%%%%%%%%%%%%%%%%%%%%%%%%%%%%%%%%%%%%%%%%%%%%%%%%%%%%%%%%%
}
