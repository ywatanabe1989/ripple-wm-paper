\noindent\hrulefill

\revTwoComment{
- In Discussion, the main findings (distance of NT bigger in hippocampus, correlation between set size and gEgR) are repeated twice.\\
}
\\
\resRevTwo{}
\authorText{
Thank you very much for your suggestions. We now separated the results and implications in different paragraphs, probably the cause of the redundancy. We hope these amendments have solved the repetitions.\\
\\
4. Discussion\\
This study hypothesizes that in low-dimensional spaces during a WM task\\
in humans, hippocampal neurons form unique NTs, primarily during SWR\\
periods. Initially, multi-unit spikes in the MTL regions were projected onto\\
three-dimensional spaces during a Sternberg task using GPFA (Figure 1D--\\
E \& Figure 2A). The NT distances across WM phases (kg F g E k, kg F g M k,\\
kg F g R k, kg E g M k, kg E g R k, and kg M g R k) were significantly larger in the hip-\\
pocampus compared to the EC and amygdala (Figure 2C--E). Also, in the\\
hippocampus, the NT distance between the encoding and retrieval phases\\
(kg F g E k) positively correlated with memory load (Figure 3C--D). The hip-\\
pocampal NT transiently expanded during SWRs (Figure 5). Lastly, the\\
hippocampal NT alternated between encoding and retrieval states, transi-\\
tioning from encoding to retrieval during SWR events (Figure 7). These\\
findings explain aspects of hippocampal neural activity during a WM task\\
in humans and offer new insights into SWRs as a state-switching element in\\
hippocampal neural states.\\
The longer disntace of NTs across the four phases in the hippocampus\\
indicates dynamic and responsive neural activity in the hippocampus dur-\\
ing the WM task. This observation corroborates previous studies indicat-\\
ing hippocampal persistent firing during the maintenance phase {[4, 5, 6, 3]}.\\
However, in the present study, applying GPFA to multi-unit activity dur-\\
ing a one-second level resolution of the WM task revealed that the NT in\\
low-dimensional space presented a memory-load dependency between the en-\\
coding and retrieval phases, denoted as kg E g R k (Figure 3). These findings\\
support the association of the hippocampus with WM processing.\\
\\
\hl{
Lines XX--XX\\
\\
REVISION\\
}
}
