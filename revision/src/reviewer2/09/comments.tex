\noindent\hrulefill

\revTwoComment{
- Fig.3c is not informative. By definition, SWR+ and SWR- will have the same duration distribution.\\
}
\\
\resRevTwo{}
\authorText{
We believe you are referring to Fig. 4c, not Fig. 3c. The purpose of Fig. 4c is not solely to display the identical distribution of SWR+ and SWR-, but also to serve as supporting evidence for the detection of SWR events, which ideally follow a log-normal distribution. To clarify these points, we have updated the legend of Fig. 4c.\\
\\
Figure 4 -- Detection of SWRs in Putative CA1 Regions\\
A. Two-dimensional UMAP {[51]} projection displays multi-unit spikes during\\
SWR + candidates (purple) and SWR $-$ candidates (yellow ). B. A cumulative\\
density plot indicates silhouette scores, reflecting UMAP clustering quality (see\\
Table 2). Hippocampal regions with silhouette scores exceeding 0.60 (equiv-\\
alent to the 75 th percentile) are identified as putative CA1 regions. SWR +\\
and SWR $-$ candidates, which were recorded from these regions, are classified\\
as SWR + and SWR $-$ respectively (ns = 1,170). C. Identical distributions\\
of SWR + (purple) and SWR $-$ (yellow ) distributions, based on their defini-\\
tions (93.0 {[65.4]} ms, median {[IQR]}). Note that these distributions exhibit log-\\
normality. D. Identical SWR incidence for both SWR + (purple) and SWR $-$\\
(yellow ), relative to the probe’s timing (mean ±95% confidence interval). How-\\
ever, 95% confidence interval may not be visibly apparent due to their narrow\\
ranges. Note that a significant SWR incidence increase was detected during\\
the initial 400 ms of the retrieval phase (0.421 {[Hz]}, *p < 0.05, bootstrap test).\\
E. Distributions of ripple band peak amplitudes for SWR $-$ (yellow ; 2.37 {[0.33]}\\
SD of baseline, median {[IQR]}) and SWR + (purple; 3.05 {[0.85]} SD of baseline,\\
median {[IQR]}) are manifested (***p < 0.001, the Brunner--Munzel test). Note\\
the log-normality for SWR + events.\\
\\
\hl{
Lines XX--XX\\
\\
REVISION\\
}
}
