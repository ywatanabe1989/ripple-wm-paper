{
\color{revision_color}
\revEditor{}
%%%%%%%%%%%%%%%%%%%%%%%%%%%%%%%%%%%%%%%%%%%%%%%%%%%%%%%%%%%%%%%%%%%%%%%%%%%%%%%%
Lines XX--XX\\

\DIFdelbegin \DIFdel{Considering these factors, this study investigates the hypothesis that hippocampal neurons exhibit unique }\DIFdelend \DIFaddbegin \DIFadd{The present study hypothesizes that human hippocampal neurons manifest low-dimensional }\DIFaddend neural trajectories (NTs) \DIFdelbegin \DIFdel{in low-dimensional space}\DIFdelend \DIFaddbegin \DIFadd{that fluctuate with WM load}\DIFaddend , particularly during SWR periods\DIFdelbegin \DIFdel{, in response to WM tasks in humans. To test }\DIFdelend \DIFaddbegin \DIFadd{. The emphasis on NTs is derived from the imperative to comprehend the continuous, dynamic representation of neurons and the facilitation of visualization and comprehension. To evaluate }\DIFaddend this hypothesis, \DIFdelbegin \DIFdel{we employed a dataset of patients performing an eight-second Sternberg task (}\DIFdelend \DIFaddbegin \DIFadd{a human WM dataset characterized by high temporal resolution --- }\DIFaddend 1 s for fixation, 2 s for encoding, 3 s for maintenance, and 2 s for retrieval \DIFdelbegin \DIFdel{) with high temporal resolution.
Intracranial electroencephalography (iEEG) signals within the medial temporal lobe (MTL) were recorded for these patients \mbox{%DIFAUXCMD
\cite{boran_dataset_2020}}\hspace{0pt}%DIFAUXCMD
. To investigate }\DIFdelend \DIFaddbegin \DIFadd{--- \mbox{%DIFAUXCMD
\cite{boran_dataset_2020} }\hspace{0pt}%DIFAUXCMD
was employed.
}\\

%%%%%%%%%%%%%%%%%%%%%%%%%%%%%%%%%%%%%%%%%%%%%%%%%%%%%%%%%%%%%%%%%%%%%%%%%%%%%%%%
}