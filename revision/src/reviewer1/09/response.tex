\resRevOne{}
\authorText{
Thank you for your inquiry regarding similar analyses.\\
\\
Firstly, GPFA has been employed in numerous studies related to cognitive performance \cite{yu_gaussian-process_2009, churchland_stimulus_2010, lin_functional_2011, churchland_neural_2012, ecker_state_2014, kao_single-trial_2015, gallego_neural_2017, wei_orderly_2019, kim_corticalhippocampal_2023}.\\
\\
Regarding previsous studies of WM processes, several studies have utilized this publicly available dataset \cite{boran_dataset_2020} over the past four years, demonstrating various phenomena correlated to WM performance \cite{li_functional_2023, sheybani_wake_2023, li_anteriorposterior_2022, li_thetaalpha_2024, ye_phase-amplitude_2022, cocina_spiking_2022, dimakopoulos_information_2022}.\\
\\
However, to the best of our knowledge, the current study is the first to focus on SWRs in this specific dataset, which possesses unique characteristics of high-temporal resolution of iEEG data during a WM task in humans. Therefore, our findings inherently provide new insights into the potential functional role of SWRs in WM tasks.
\\
To incorporate these backgrounds, we have updated the Introduction section as follows:\\
}