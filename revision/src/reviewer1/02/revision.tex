{
\color{revision_color}
\revEditor{}
%%%%%%%%%%%%%%%%%%%%%%%%%%%%%%%%%%%%%%%%%%%%%%%%%%%%%%%%%%%%%%%%%%%%%%%%%%%%%%%%

% \section{Introduction}
Working memory (WM)\DIFdelbegin \DIFdel{is crucial in everyday life; however, its neural mechanism has yet to be fully elucidated. Specifically, the hippocampus's involvement in WM processing, a pivotal region for memory, is the subject of ongoing research \mbox{%DIFAUXCMD
\cite{scoville_loss_1957} }\hspace{0pt}%DIFAUXCMD
\mbox{%DIFAUXCMD
\cite{squire_legacy_2009}  }\hspace{0pt}%DIFAUXCMD
\mbox{%DIFAUXCMD
\cite{boran_persistent_2019} }\hspace{0pt}%DIFAUXCMD
\mbox{%DIFAUXCMD
\cite{kaminski_persistently_2017} }\hspace{0pt}%DIFAUXCMD
\mbox{%DIFAUXCMD
\cite{kornblith_persistent_2017} }\hspace{0pt}%DIFAUXCMD
\mbox{%DIFAUXCMD
\cite{faraut_dataset_2018} }\hspace{0pt}%DIFAUXCMD
\mbox{%DIFAUXCMD
\cite{borders_hippocampus_2022} }\hspace{0pt}%DIFAUXCMD
\mbox{%DIFAUXCMD
\cite{li_functional_2023} }\hspace{0pt}%DIFAUXCMD
\mbox{%DIFAUXCMD
\cite{dimakopoulos_information_2022}}\hspace{0pt}%DIFAUXCMD
. Understanding the hippocampus ’ role in working memory is instrumental in deepening our comprehension of cognitive processes and could potentially enhance cognitive abilities}\DIFdelend \DIFaddbegin \DIFadd{, serving as a key player in cognitive abilities, underpins our daily activities and relations with the world, from basic perceptual decision making to sophisticated cognitive operations. One remarkable component in the neural mechanisms of WM is the hippocampus, an area identified as being crucial for various forms of memory \mbox{%DIFAUXCMD
\cite{scoville_loss_1957, squire_legacy_2009, boran_persistent_2019, kaminski_persistently_2017, kornblith_persistent_2017, faraut_dataset_2018, borders_hippocampus_2022, li_functional_2023, dimakopoulos_information_2022}}\hspace{0pt}%DIFAUXCMD
. Unraveling the role and contributions of the hippocampus within the realm of WM informs our understanding of the cognitive dynamics underpinning everyday functionality. This knowledge may ultimately foster the enhancement of cognitive performance and the development of interventions for memory-related disorders}\DIFaddend .
\\
\indent
\DIFdelbegin \DIFdel{Current evidence suggests that a }\DIFdelend \DIFaddbegin \DIFadd{During WM tasks, hippocampal networks yield }\DIFaddend transient, synchronized \DIFdelbegin \DIFdel{oscillation, called }\DIFdelend \DIFaddbegin \DIFadd{oscillations known as }\DIFaddend sharp-wave ripples (SWRs)\DIFdelbegin \DIFdel{\mbox{%DIFAUXCMD
\cite{buzsaki_hippocampal_2015}}\hspace{0pt}%DIFAUXCMD
, is associated with several cognitive functions. These include memory replay \mbox{%DIFAUXCMD
\cite{wilson_reactivation_1994} }\hspace{0pt}%DIFAUXCMD
\mbox{%DIFAUXCMD
\cite{nadasdy_replay_1999} }\hspace{0pt}%DIFAUXCMD
\mbox{%DIFAUXCMD
\cite{lee_memory_2002} }\hspace{0pt}%DIFAUXCMD
\mbox{%DIFAUXCMD
\cite{diba_forward_2007} }\hspace{0pt}%DIFAUXCMD
\mbox{%DIFAUXCMD
\cite{davidson_hippocampal_2009}}\hspace{0pt}%DIFAUXCMD
, memory consolidation \mbox{%DIFAUXCMD
\cite{girardeau_selective_2009} }\hspace{0pt}%DIFAUXCMD
\mbox{%DIFAUXCMD
\cite{ego-stengel_disruption_2010} }\hspace{0pt}%DIFAUXCMD
\mbox{%DIFAUXCMD
\cite{fernandez-ruiz_long-duration_2019} }\hspace{0pt}%DIFAUXCMD
\mbox{%DIFAUXCMD
\cite{kim_corticalhippocampal_2022}}\hspace{0pt}%DIFAUXCMD
, memory recall \mbox{%DIFAUXCMD
\cite{wu_hippocampal_2017} }\hspace{0pt}%DIFAUXCMD
\mbox{%DIFAUXCMD
\cite{norman_hippocampal_2019} }\hspace{0pt}%DIFAUXCMD
\mbox{%DIFAUXCMD
\cite{norman_hippocampal_2021}}\hspace{0pt}%DIFAUXCMD
, and neural plasticity \mbox{%DIFAUXCMD
\cite{behrens_induction_2005} }\hspace{0pt}%DIFAUXCMD
\mbox{%DIFAUXCMD
\cite{norimoto_hippocampal_2018}}\hspace{0pt}%DIFAUXCMD
. These associations suggest that SWR may be a fundamental computational manifestation of hippocampal processing, contributing to working memory performance as well .
}%DIFDELCMD < \\
%DIFDELCMD < \indent
%DIFDELCMD < %%%
\DIFdel{Recent studies have found that low-dimensional representations in hippocampal neurons can explain WM task performances. Specifically, the firing patterns of place cells \mbox{%DIFAUXCMD
\cite{okeefe_hippocampus_1971} }\hspace{0pt}%DIFAUXCMD
\mbox{%DIFAUXCMD
\cite{okeefe_place_1976} }\hspace{0pt}%DIFAUXCMD
\mbox{%DIFAUXCMD
\cite{ekstrom_cellular_2003} }\hspace{0pt}%DIFAUXCMD
\mbox{%DIFAUXCMD
\cite{kjelstrup_finite_2008} }\hspace{0pt}%DIFAUXCMD
\mbox{%DIFAUXCMD
\cite{harvey_intracellular_2009}}\hspace{0pt}%DIFAUXCMD
, found in the hippocampus , have been identified within dynamic, nonlinear three-dimensional hyperbolic spaces in rats \mbox{%DIFAUXCMD
\cite{zhang_hippocampal_2022}}\hspace{0pt}%DIFAUXCMD
. Additionally, grid cells in the entorhinal cortex (EC), which is the main pathway to the hippocampus \mbox{%DIFAUXCMD
\cite{naber_reciprocal_2001} }\hspace{0pt}%DIFAUXCMD
\mbox{%DIFAUXCMD
\cite{van_strien_anatomy_2009} }\hspace{0pt}%DIFAUXCMD
\mbox{%DIFAUXCMD
\cite{strange_functional_2014}}\hspace{0pt}%DIFAUXCMD
, exhibited a toroidal geometry during exploration in rats \mbox{%DIFAUXCMD
\cite{gardner_toroidal_2022}}\hspace{0pt}%DIFAUXCMD
}\DIFdelend \DIFaddbegin \DIFadd{, which have been found to replay sequences of recent and prospective memory traces \mbox{%DIFAUXCMD
\cite{foster_reverse_2006, karlsson_awake_2009, carr_hippocampal_2011, pfeiffer_hippocampal_2013}}\hspace{0pt}%DIFAUXCMD
. Moreover, functional correlations between awake SWRs and WM performance over multi-day scales have been elucidated by selective SWR suppressions \mbox{%DIFAUXCMD
\cite{girardeau_selective_2009, jadhav_awake_2012, singer_hippocampal_2013}}\hspace{0pt}%DIFAUXCMD
, and prolongation events \mbox{%DIFAUXCMD
\cite{fernandez-ruiz_long-duration_2019}}\hspace{0pt}%DIFAUXCMD
, as well as functional lesioning in the dentate gyrus, a subregion of the hippocampus \mbox{%DIFAUXCMD
\cite{sasaki_dentate_2018}}\hspace{0pt}%DIFAUXCMD
. Certain studies have emphasized the coordination of SWRs with other oscillation components in the facilitation of WM processing \mbox{%DIFAUXCMD
\cite{tamura_hippocampal-prefrontal_2017, daume_control_2024}}\hspace{0pt}%DIFAUXCMD
. Furthermore, SWR events that occur seconds before memory recall proffer a notion of their essentiality for effective execution of WM tasks \mbox{%DIFAUXCMD
\cite{wu_hippocampal_2017, norman_hippocampal_2019, norman_hippocampal_2021}}\hspace{0pt}%DIFAUXCMD
. Despite these preliminary insights, our comprehensive understanding of SWRs and their temporal relationship with WM processes remains largely incomplete}\DIFaddend .
\\
\indent
\DIFdelbegin \DIFdel{However, these studies primarily focus on spatial navigation in rodents, which poses limitations. To illustrate, }\DIFdelend \DIFaddbegin \DIFadd{One noteworthy limitation in the current body of research is predicated on the use of rodent navigation tasks, wherein the temporal attributes of }\DIFaddend the \DIFdelbegin \DIFdel{temporal resolution of navigation tasks is inadequate as the timing }\DIFdelend \DIFaddbegin \DIFadd{task were not sufficiently granular to discern the exact timings }\DIFaddend of memory acquisition\DIFdelbegin \DIFdel{and recall is not clearly delineated. Consequently, there is a relative paucity of research on the impact of SWRs on WM performance \mbox{%DIFAUXCMD
\cite{jadhav_awake_2012}}\hspace{0pt}%DIFAUXCMD
. Further, the presence of noise in signals recorded during rodent movement complicates the detection of SWRs \mbox{%DIFAUXCMD
\cite{Watanabe_2021}}\hspace{0pt}%DIFAUXCMD
. Therefore, to clarify }\DIFdelend \DIFaddbegin \DIFadd{, retrieval, and decision-making processes. Furthermore, the detection of SWRs during predominantly immobile periods in rodents \mbox{%DIFAUXCMD
\cite{foster_reverse_2006, karlsson_awake_2009, carr_hippocampal_2011, pfeiffer_hippocampal_2013, jadhav_awake_2012, singer_hippocampal_2013, fernandez-ruiz_long-duration_2019}}\hspace{0pt}%DIFAUXCMD
, likely due to potential contamination from electromyographic noise \mbox{%DIFAUXCMD
\cite{Watanabe_2021}}\hspace{0pt}%DIFAUXCMD
, limits the extrapolation of experimental findings to subjectively equivalent functionality of WM in humans. It is thus imperative to explore }\DIFaddend the relationship between SWRs and \DIFdelbegin \DIFdel{WM tasks, research with human subjects is necessary}\DIFdelend \DIFaddbegin \DIFadd{human WM in a noise-controlled experimental setup with a temporally precise measurement of WM events}\DIFaddend .
\\
\indent
\DIFdelbegin \DIFdel{Considering these factors, this study investigates the hypothesis that hippocampal neurons exhibit unique }\DIFdelend \DIFaddbegin \DIFadd{The present study hypothesizes that human hippocampal neurons manifest low-dimensional }\DIFaddend neural trajectories (NTs) \DIFdelbegin \DIFdel{in low-dimensional space}\DIFdelend \DIFaddbegin \DIFadd{that fluctuate with WM load}\DIFaddend , particularly during SWR periods\DIFdelbegin \DIFdel{, in response to WM tasks in humans. To test }\DIFdelend \DIFaddbegin \DIFadd{. The emphasis on NTs is derived from the imperative to comprehend the continuous, dynamic representation of neurons and the facilitation of visualization and comprehension. To evaluate }\DIFaddend this hypothesis, \DIFdelbegin \DIFdel{we employed a dataset of patients performing an eight-second Sternberg task (}\DIFdelend \DIFaddbegin \DIFadd{a human WM dataset characterized by high temporal resolution --- }\DIFaddend 1 s for fixation, 2 s for encoding, 3 s for maintenance, and 2 s for retrieval \DIFdelbegin \DIFdel{) with high temporal resolution.
Intracranial electroencephalography (iEEG) signals within the medial temporal lobe (MTL) were recorded for these patients \mbox{%DIFAUXCMD
\cite{boran_dataset_2020}}\hspace{0pt}%DIFAUXCMD
. To investigate }\DIFdelend \DIFaddbegin \DIFadd{--- \mbox{%DIFAUXCMD
\cite{boran_dataset_2020} }\hspace{0pt}%DIFAUXCMD
was employed.
}\\
\indent
\DIFadd{To aptly analyze }\DIFaddend low-dimensional \DIFdelbegin \DIFdel{NTs, we utilized }\DIFdelend \DIFaddbegin \DIFadd{dynamics, dimensionality reduction techniques were implemented. Interestingly, recent findings have shown the presence of dynamic, nonlinear low-dimensional spaces within the firing patterns of hippocampal \mbox{%DIFAUXCMD
\cite{zhang_hippocampal_2022} }\hspace{0pt}%DIFAUXCMD
and entorhinal cortex (EC) neurons in rodents \mbox{%DIFAUXCMD
\cite{gardner_toroidal_2022}}\hspace{0pt}%DIFAUXCMD
. Accordingly, we applied }\DIFaddend Gaussian-process factor analysis (GPFA), \DIFdelbegin \DIFdel{an established method for analyzing }\DIFdelend \DIFaddbegin \DIFadd{a proven methodology for }\DIFaddend neural population dynamics \DIFdelbegin \DIFdel{\mbox{%DIFAUXCMD
\cite{yu_gaussian-process_2009}}\hspace{0pt}%DIFAUXCMD
}\DIFdelend \DIFaddbegin \DIFadd{analyses \mbox{%DIFAUXCMD
\cite{yu_gaussian-process_2009, churchland_stimulus_2010, lin_functional_2011, churchland_neural_2012, ecker_state_2014, kao_single-trial_2015, gallego_neural_2017, wei_orderly_2019, kim_corticalhippocampal_2023}}\hspace{0pt}%DIFAUXCMD
, on intracranial electroencephalography (iEEG) signals recorded from the medial temporal lobe, including the hippocampus, of patients performing a structured WM task}\DIFaddend .
% \label{sec:introduction}

%%%%%%%%%%%%%%%%%%%%%%%%%%%%%%%%%%%%%%%%%%%%%%%%%%%%%%%%%%%%%%%%%%%%%%%%%%%%%%%%
}