\resRevOne{}
\authorText{
Thank you for your insightful question regarding the frequency range used to define SWRs in our study.
\\
\\
The referenced paper (doi:https://doi.org/10.1038/nn1571, Nature, 2005) investigated stimulus-induced and spontaneous SWRs in rats. However, in humans, researchers have observed a more pronounced power increase in the slower frequency band, such as 80--140 Hz, compared to that of rodents (150--250 Hz). They postulated that oscillations in this frequency range correspond to SWRs in other species, including rodents. This study adheres to this perspective.\\
\\
In fact, a concensus paper in the field related to hippocampal SWRs \cite{liu_consensus_2022} mentions as follows:\\
\\
``SPW-R frequency band criterion for rodents (100 to 250 Hz) is generally higher than for monkeys (95 to 250 Hz) or humans (70--250 Hz, most use 80--150 Hz bandpass filters; Supplementary Table S1).''\\
}