\resRevOne{}
\authorText{
Thank you for your inquiry regarding the application of Gaussian-Process Factor Analysis (GPFA) in studying working memory and other cognitive functions. We have thoroughly reviewed the literature and identified multiple instances where GPFA has been effectively utilized in cognitive research.\\
\\
GPFA was first introduced by Yu et al., in 2009 \cite{yu_gaussian-process_2009}, a study that has been cited extensively, indicating its importance in neural data analysis. This foundational paper provides intuitive visualizations on how GPFA extracts neural trajectories in its Figure 2 and 8.\\
\\
Also, the elephant package, which we utilized for the GPFA calculations, provides a comprehensive and practical tutorial: \url{https://elephant.readthedocs.io/en/latest/tutorials/gpfa.html}.\\
\\
We have incorporated the following key studies into our manuscript that demonstrate the versatility and effectiveness of GPFA in analyzing neural dynamics across various contexts:\\
\\
Yu, B. M. et al. (2009): GPFA was used to extract dynamic patterns from neurons' spike train data in single trials.\\
\\
Churchland, M. M. et al. (2010): GPFA demonstrated reduced neural variability upon stimulus onset, indicating synchronized cortical responses.\\
\\
Lin, D. et al. (2011): GPFA identified specific neural activity patterns linked to aggression in mice.\\
\\
Churchland, M. et al. (2012): During reach tasks, GPFA revealed dynamic neural patterns suggesting continuous evolution of neural state.\\
\\
Ecker, A. S. et al. (2014): GPFA analyzed noise correlations within macaque visual cortex, showing how external states influence internal dynamics.\\
\\
Kao, J. C. et al. (2015): Demonstrated GPFA's application in decoding neural dynamics for brain-machine interface improvements.\\
\\
Gallego, J. A. et al. (2017): Used GPFA to map motor cortex activity onto a lower-dimensional manifold.\\
\\
Wei, Z. et al. (2019): GPFA revealed how population dynamics in premotor cortex are organized on a trial-by-trial basis.\\
\\
Kim, J. et al. (2023): Explored the dynamics of cortical-hippocampal interactions during motor tasks, indicating complex coordination.\\
\\
These references underscore GPFA’s significant role in advancing our understanding of neural mechanisms underlying cognitive functions, including working memory.\\
}
