{
\color{revision_color}
\revEditor{}
%%%%%%%%%%%%%%%%%%%%%%%%%%%%%%%%%%%%%%%%%%%%%%%%%%%%%%%%%%%%%%%%%%%%%%%%%%%%%%%%

\subsection{Detection of Hippocampal SWRs from Putative CA1 Regions}
\DIFdelbegin \DIFdel{To enhance the precision }\DIFdelend \DIFaddbegin \DIFadd{The precise localization of recording electrodes within the hippocampus presents significant challenges in human studies, primarily due to the typical unavailability of postmortem histological confirmation. To improve the accuracy }\DIFaddend of recording sites and SWR detection \DIFdelbegin \DIFdel{, we approximated the electrode placements in the }\DIFdelend \DIFaddbegin \DIFadd{in the hippocampus, we established an inclusion criterion that required the electrode to be in the putative }\DIFaddend CA1 regions of the hippocampus\DIFdelbegin \DIFdel{using distinguished }\DIFdelend \DIFaddbegin \DIFadd{. This was based on distinct }\DIFaddend multi-unit spike patterns \DIFaddbegin \DIFadd{observed }\DIFaddend during SWR events \DIFaddbegin \DIFadd{compared to baseline periods}\DIFaddend . SWR$^+$/SWR$^-$ candidates from each session and hippocampal region were embedded in two-dimensional space using UMAP (Figure~\ref{fig:04}A).\footnote{Consider the AHL in session \#1 of subject \#1 as a case in point.} With the silhouette score as a quality metric for clustering (Figure~\ref{fig:04}B and Table~\ref{tab:02}), recording sites demonstrating an average silhouette score exceeding 0.6 across all sessions were identified as putative CA1 regions.\footnote{The identified regions were the AHL of subject \#1, AHR of subject \#3, PHL of subject \#4, AHL of subject \#6, and AHR of subject \#9.} (Tables~\ref{tab:02} and \ref{tab:03}). We identified five putative CA1 regions, four of which were not indicated as seizure onset zones (Table~\ref{tab:01}).

%%%%%%%%%%%%%%%%%%%%%%%%%%%%%%%%%%%%%%%%%%%%%%%%%%%%%%%%%%%%%%%%%%%%%%%%%%%%%%%%
}