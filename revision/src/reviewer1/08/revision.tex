{
\color{revision_color}
\revEditor{}
%%%%%%%%%%%%%%%%%%%%%%%%%%%%%%%%%%%%%%%%%%%%%%%%%%%%%%%%%%%%%%%%%%%%%%%%%%%%%%%%

\DIFadd{Furthermore}\DIFaddend , in the present study, \DIFdelbegin \DIFdel{applying }\DIFdelend \DIFaddbegin \DIFadd{the application of }\DIFaddend GPFA to multi-unit activity during a one-second level resolution of the WM task revealed that the \DIFdelbegin \DIFdel{neural }\DIFdelend NT in low-dimensional space presented a memory-load dependency between the encoding and retrieval phases, denoted as $\mathrm{\lVert g_{E}g_{R} \rVert}$ (Figure~\ref{fig:03}). \DIFdelbegin \DIFdel{These findings support }\DIFdelend \DIFaddbegin \DIFadd{Notably, this dependency was not identified in other phase combinations. These observations suggest that the encoding and retrieval states expanded in opposite directions in response to the WM load. Overall, these findings not only reinforce }\DIFaddend the association of the hippocampus with WM processing \DIFaddbegin \DIFadd{in humans but also introduce the phenomenon of "hippocampal neural fluctuation between encoding and retrieval states"}\DIFaddend .

%%%%%%%%%%%%%%%%%%%%%%%%%%%%%%%%%%%%%%%%%%%%%%%%%%%%%%%%%%%%%%%%%%%%%%%%%%%%%%%%
}