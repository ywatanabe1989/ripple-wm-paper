{
\color{revision_color}
\revEditor{}
%%%%%%%%%%%%%%%%%%%%%%%%%%%%%%%%%%%%%%%%%%%%%%%%%%%%%%%%%%%%%%%%%%%%%%%%%%%%%%%%
Lines XX--XX\\

\subsection{Identifying SWR candidates from hippocampal regions}
Potential SWR events within the hippocampus were detected using a widely used method \cite{liu_consensus_2022}. LFP signals from a region of interest (ROI) like AHL, were re-referenced by deducting the averaged signal from locations outside the ROI (for instance, AHR, PHL, PHR, ECL, ECR, AL, and AR). The re-referenced LFP signals were then filtered with a ripple-band filter (80--140 Hz) to determine SWR candidates, marked as $\textrm{SWR}^+$ candidates. SWR detection was carried out using a published tool (\url{https://github.com/Eden-Kramer-Lab/ripple_detection}) \cite{kay_hippocampal_2016}, with the bandpass range adjusted to 80--140 Hz for humans \DIFdelbegin \DIFdel{\mbox{%DIFAUXCMD
\cite{norman_hippocampal_2019} }\hspace{0pt}%DIFAUXCMD
\mbox{%DIFAUXCMD
\cite{norman_hippocampal_2021}}\hspace{0pt}%DIFAUXCMD
}\DIFdelend \DIFaddbegin \DIFadd{\mbox{%DIFAUXCMD
\cite{norman_hippocampal_2019, norman_hippocampal_2021, liu_consensus_2022}}\hspace{0pt}%DIFAUXCMD
}\DIFaddend , unlike the \DIFdelbegin \DIFdel{initial }\DIFdelend \DIFaddbegin \DIFadd{original }\DIFaddend 150--250 Hz range typically applied to rodents \DIFaddbegin \DIFadd{\mbox{%DIFAUXCMD
\cite{foster_reverse_2006, karlsson_awake_2009, carr_hippocampal_2011, pfeiffer_hippocampal_2013, jadhav_awake_2012, singer_hippocampal_2013, buzsaki_hippocampal_2015, wu_hippocampal_2017, fernandez-ruiz_long-duration_2019}}\hspace{0pt}%DIFAUXCMD
}\DIFaddend .
}


%%%%%%%%%%%%%%%%%%%%%%%%%%%%%%%%%%%%%%%%%%%%%%%%%%%%%%%%%%%%%%%%%%%%%%%%%%%%%%%%

% SWR detection was carried out using a published tool (https://github.\\
% com/Eden-Kramer-Lab/ripple\_detection) {[50]}, with the bandpass range\\
% adjusted to 80--140 Hz for humans {[21, 22, 49]}, unlike the original 150--250\\
% Hz range typically applied to rodents {[10]}.\\
% \\
% Reference:\\
% {[21]} Y. Norman, E. M. Yeagle, S. Khuvis, M. Harel, A. D. Mehta, R. Malach,\\
% Hippocampal sharp-wave ripples linked to visual episodic recollection in\\
% humans, Science 365 (6454) (2019) eaax1030. doi:10.1126/science.\\
% aax1030.\\
% URL https://www.sciencemag.org/lookup/doi/10.1126/science.\\
% aax1030\\
% \\
% {[22]} Y. Norman, O. Raccah, S. Liu, J. Parvizi, R. Malach, Hippocampal\\
% ripples and their coordinated dialogue with the default mode network\\
% during recent and remote recollection, Neuron 109 (17) (2021) 2767--\\
% 2780.e5, publisher: Elsevier. doi:10.1016/j.neuron.2021.06.020.\\
% URL\\
% https://www.cell.com/neuron/abstract/S0896-6273(21)\\
% \\
% {[49]} A. A. Liu, S. Henin, S. Abbaspoor, A. Bragin, E. A. Buffalo, J. S. Farrell,\\
% D. J. Foster, L. M. Frank, T. Gedankien, J. Gotman, J. A. Guidera, K. L.\\
% Hoffman, J. Jacobs, M. J. Kahana, L. Li, Z. Liao, J. J. Lin, A. Losonczy,\\
% R. Malach, M. A. van der Meer, K. McClain, B. L. McNaughton, Y. Nor-\\
% man, A. Navas-Olive, L. M. de la Prida, J. W. Rueckemann, J. J. Sakon,\\
% I. Skelin, I. Soltesz, B. P. Staresina, S. A. Weiss, M. A. Wilson, K. A.\\
% Zaghloul, M. Zugaro, G. Buzsáki, A consensus statement on detection of\\
% hippocampal sharp wave ripples and differentiation from other fast os-\\
% cillations, Nature Communications 13 (1) (2022) 6000, number: 1 Pub-\\
% lisher: Nature Publishing Group. doi:10.1038/s41467-022-33536-x.\\
% URL https://www.nature.com/articles/s41467-022-33536-x\\
% \\
% {[50]} K. Kay, M. Sosa, J. E. Chung, M. P. Karlsson, M. C. Larkin, L. M.\\
% Frank, A hippocampal network for spatial coding during immobility and\\
% }
% sleep, Nature 531 (7593) (2016) 185--190. doi:10.1038/nature17144.\\
